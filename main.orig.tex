\documentclass[aspectratio=169, xcolor=table, 10pt]{beamer}
\usepackage[serbian]{babel}
\usepackage[OT2, T1]{fontenc}
\usepackage[utf8]{inputenc}
\usepackage{amsmath}
\usepackage{hyperref}
\usepackage{amsfonts}
\usepackage{amssymb}
\usepackage{amsthm}
\usepackage{fancyhdr}
\usepackage{extramarks}
\usepackage{graphicx}
\usepackage{tikz}
\usepackage{cmsrb}
\usepackage{tikz}
\usepackage{subfigure}
\usetheme{Madrid}
\usecolortheme{crane}


\setbeamercolor{table even}{bg=yellow} % Yellow for even rows
\setbeamercolor{table odd}{bg=yellow}  % Yellow for odd rows
\setbeamercolor{last row}{bg=orange}   % Orange for the last row

\theoremstyle{definition}
\newtheorem{defi}{Definicija}[section]

\newtheorem{prim}{Primer}[section]



\AtBeginSection[]
{
  \begin{frame}
    \frametitle{Sadržaj}
    \tableofcontents[currentsection]
  \end{frame}
}


\title{Milenijumski problem - P protiv NP}
\author{An\dj ela Milovanovi\' c}
\institute[MATF]{Matematički fakultet u Beogradu}
\date{Septembar 2024}

\begin{document}

\maketitle

\section{Uvod}

\begin{frame}{Teorija Izra\v cunjljivosti}

    \begin{itemize}
        \item Teorija je nastala tokom tridesetih godina prošlog veka, u vreme kada klasični računari nisu postojali.
        \item Pioniri ove nauke bili su Alonzo Čerč, Alan Tjuring, Emil Post, Kurt Gedel i drugi.
        \item Prvi radovi na teoriji izračunljivosti bili su inspirisani čuvenim Gedelovim dokazima teoreme o nekompletnosti.
        \item U ovom periodu definisan je koncept koji liči na današnje algoritme (“efektivne procedure”).
        \item Glavno pitanje kojim se ova teorija bavi: “Šta se sve može uraditi pomoću računara”?
        \item U nedostatku pravog kompjutera, za izvođenje dokaza i demonstraciju rešenja problema često je korišćena Tjuringova mašina, jednostavan uređaj koji je u stanju da izvrši bilo koji algoritam, baš kao i današnji računari.
    \end{itemize}
    
\end{frame}

\begin{frame}{Formalni jezici}

\begin{defi}
    \begin {itemize}
    \item Azbuka $\Sigma$ je bilo koji neprazan skup simbola.
    \item Re\v c nad azbukom $\Sigma$ je bilo koji kona\v can niz simbola azbuke $\Sigma$.
    \item Skup $\Sigma^{*}$ je skup svih re\v ci nad azbukom $\Sigma$.
    \item Jezik nad azbukom $\Sigma$ je bilo koji podskup skupa $\Sigma^{*}$.
    \end {itemize}
\end{defi}

\begin{prim}
    \begin {itemize}
    \item Posmatrajmo azbuku $\Sigma = \{ a, b, c \}$.
    \item Tada je $w = abba$ re\v c nad ovom azbukom.
    \item Primeri jezika nad azbukom $\Sigma$ su: $$L_1 = \varnothing \quad L_2 = \{ a, ab, bc, ca \} \quad L_3 = \{ \epsilon, a, aa, aaa, \dots \} \quad L_4 = \Sigma^{*}.$$
    \end {itemize}
\end{prim}
    
\end{frame}

\begin{frame}{Hijerarhija \v Comskog}

\begin{table}[ht]
\centering
\rowcolors{1}{yellow}{yellow}
\begin{tabular}{|c|c|c|}
\hline
\textbf{Gramatika} & \textbf{Tip jezika} & \textbf{Automat} \\ 
\hline
Tip 3 & Regularni jezici         & Kona\v cni automat \\ 
\hline
Tip 2 & Kontekstno slobodni jezici & Potisni Automat \\ 
\hline
Tip 1 & Kontekstno-osetljivi jezici & Linearno ograni\v ceni automat\\ 
\hline
\rowcolor{orange} Tip 0 & Rekurzivno prebrojivi jezici & Tjuringova ma\v sina\\ 
\hline
\end{tabular}
\end{table}

\begin{figure}
    \centering
    \includegraphics[width=0.30\linewidth]{Chomsky-hierarchy.png}
    \caption{Hijerarhija \v Comskog}
    \label{fig:enter-label}
\end{figure}

\end{frame}

\section{Tjuringova ma\v sina}

\begin{frame}{Tjuringova ma\v sina (neformalno)}

\begin{itemize}
\item Mašina koristi beskonačnu traku podeljenu u diskretne ćelije.

\item Svaka ćelija može da sadrži neki od simbola iz alfabeta koji mašina koristi (slova, brojevi).

\item Mašina ima “glavu” koja je u datom trenutku pozicionirana iznad neke ćelije.

\item Mašina ima svoje “stanje”, jedno iz predefinisanog seta mogućih stanja.

\item Glava može da pročita simbol koji se nalazi na traci neposredno ispod nje.

\item Na bazi internog stanja, pročitanog simbola i predefinisanih pravila uskladištenih u mašini:

\begin {itemize}

\item Glava mašine u ćeliju ispod nje, upisuje neki drugi simbol ili ponovo upisuje isti.

\item Mašina menja svoje interno stanje ili zadržava postojeće. Jedno od njih zaustavlja rad mašine.

\item Mašina pomera glavu levo ili desno duž beskonačne trake na novu ćeliju.



\end{itemize}

\item Tipična instrukcija: Ako je mašina u stanju A i ako je pročitan simbol 1, u ćeliju upisati simbol 0, promeniti stanje mašine u B i pomeriti glavu u desnu stranu.

\item U narednom slajdu dajemo formalnu definiciju Tjuringove ma\v sine (Videti \cite{hopcroft}).
\end{itemize}
\end{frame}

\begin{frame}{Tjuringova ma\v sina (formalno)}

\begin{defi}
    Tjuringova ma\v sina je ure\dj ena sedmorka: $M = \langle Q, \Gamma, b, \Sigma, \delta, q_0, F \rangle$ tako da va\v zi:
    \begin{itemize}
        \item $Q$ je konačan, neprazan skup stanja.
        \item $\Gamma$ je konačan, neprazan skup simbola trake.
        \item $b\in \Gamma$ je prazan simbol (jedini koji mo\v ze da se javi na traci beskonačno mnogo puta).
        \item $\Sigma \subseteq \Gamma \setminus \{b\}$ je skup ulaznih simbola.
        \item $\delta :(Q\setminus F)\times \Gamma \not\rightarrow Q\times \Gamma \times \{L,R\}$ je parcijalna funkcija tranzicije, gde je L pomeraj ulevo,a R pomeraj udesno. Ako $\delta$ nije definisana u trenutnom stanju i za trenutni simbol trake, onda se ma\v sina zaustavlja (haltuje).
        \item $ q_{0}\in Q$ je početno stanje.
        \item $F\subseteq Q$ je skup krajnjih stanja. Ka\v zemo da ma\v sina prihvata po\v cetnu nisku trake ako se zaustavi (haltuje) u nekom stanju iz $F$.
    \end{itemize}
    
\end{defi}
    
\end{frame}

\begin{frame}{Tjuringova ma\v sina}

\begin{figure}[ht]
    \centering
    \subfigure[Tjuringova ma\v sina - ilustraciona maketa]{
    \includegraphics[width=0.45\linewidth]{Turing_Machine.jpg}
    }
    \hfill
    \subfigure[Tjuringova ma\v sina - definicija preko grafa]{
    \includegraphics[width=0.45\linewidth]{Turing Machine - graph.png}
    }
    
\end{figure}
    
\end{frame}

\begin{frame}{Tjuringova ma\v sina - ekvivalencija sa modernim ra\v cunarima}

\begin{itemize}
    \item Tjuringova ma\v sina je u stanju da re\v si svaki problem koji je mogu\' ce re\v siti kori\v s\' cenjem klasi\v cnih kompjutera.
    \item Tjuringova ma\v sina formalizuje koncept algoritma kao i slo\v zenosti algoritma (broj koraka koji je potreban Tjuringovoj ma\v sini da prihvati nisku).
    \item Pojam “Turing-Complete” se vezuje za programske jezike koji imaju istu ekspresivnost kao Tjuringove ma\v sine (C, C++, Java, Python i dr.).
    \item \v Cak je i \LaTeX \ “Turing-Complete”. U teoriji, svaki C program je mogu\' ce napisati i u \LaTeX-u.
\end{itemize}
    
\end{frame}

\section{Klase Slo\v zenosti}

\begin{frame}{Problemi odlučivosti i funkcijski problemi}
\begin{itemize}

\item Postoje dve vrste problema koje Tjuringova mašina može da rešava:

\begin{itemize}
\item Problemi odlučivosti: U pitanju su problemi na koje se može odgovoriti samo sa “da” ili “ne”.

\item Funkcijski problemi: U ovim problemima Tjuringova mašina treba da generiše izlaz koji predstavlja rezultat primene određene funkcije na ulazne podatke. 
\end{itemize}

\item Praktično svi značajni kompjuterski problemi mogu se formulisati na oba načina: i kao problem odlučivosti, i kao funkcijski problem.

\item Primer problema odlučivosti:

\begin{itemize}
\item Da li broj  240.996.255.422.547 ima prost faktor manji od 15.000.000?

\item Da li postoji zatvorena putanja koja obilazi sve srpske gradove sa više od 5.000 stanovnika a kraća je od 2.000 kilometara?
\end{itemize}

\item Funkcionalna, uopštenija verzija prethodna dva problema (koja je obično teža za rešavanje):
\begin{itemize}
\item Naći proste činioce broja 240.996.255.422.547.

\item Kolika je dužina najkraće zatvorene putanje koja obilazi sve evropske glavne gradove?
\end{itemize}

\item \textbf{Nadalje ćemo se baviti isključivo problemima odlučivosti}.

\end{itemize}
\end{frame}

\begin{frame}{Klasa slo\v zenosti $\mathcal{P}$}

\begin{itemize}
    
\item Nadalje problem odlu\v civosti zapravo postaje problem pripadanja niske jeziku. Jezici \v cije niske uvek prihavata neka Tjuringova ma\v sina se nazivaju rekurzivno prebrojivi.
\item Zapravo, u nastakvu re\v c problem treba poistovetiti sa re\v cju jezik, te \' cemo sad dati definiciju $\mathcal{P}$ klase jezika.

\end{itemize}

\begin{defi}
    Ako za deterministi\v cku Tjuringovu ma\v sinu $M$ postoji polinom $p(n)$ tako da za svaku nisku du\v zine $n$, ma\v sina $M$ ne napravi vi\v se od $p(n)$ koraka, ka\v zemo da je $M$ \textbf{polinomska deterministi\v cka Tjuringova ma\v sina}.

    Klasa $\mathcal{P}$ je skup svih jezika za koje postoji polinomska deterministi\v cka Tjuringova ma\v sina koja ih prihvata.
\end{defi}

\begin{itemize}
    \item Drugim re\v cima, $\mathcal{P}$ je klasa \textbf{problema} za koje postoji algoritam polinomske slo\v zenosti koji se mo\v ze izvr\v siti na klasi\v cnom ra\v cunaru.
\end{itemize}
    
\end{frame}

\begin{frame}{Klasa slo\v zenosti $\mathcal{NP}$}

\begin{defi}
    Ako za nedeterministi\v cku Tjuringovu ma\v sinu $M$ postoji polinom $p(n)$ tako da za svaku nisku du\v zine $n$, ma\v sina $M$ ne napravi vi\v se od $p(n)$ koraka, ka\v zemo da je $M$ \textbf{polinomska nedeterministi\v cka Tjuringova ma\v sina}.

    Klasa $\mathcal{NP}$ je skup svih jezika za koje postoji polinomska nedeterministi\v cka Tjuringova ma\v sina koja ih prihvata.
\end{defi}

\begin{itemize}
    \item Kako su sve deterministi\v cke ma\v sine podskup nedeterministi\v ckih, to je $$\mathcal{P} \subseteq \mathcal{NP}.$$
\end{itemize}
\end{frame}

\begin{frame}{$\mathcal{P}$ vs. $\mathcal{NP}$}
    \begin{figure}[ht]
        \centering
        \includegraphics[width=0.60\linewidth]{P vs NP.png}
        \caption{$\mathcal{P}$ vs. $\mathcal{NP}$ - dijagram}
    \end{figure}
\end{frame}

\begin{frame}{$\mathcal{NP}$-complete klasa problema}

\begin{itemize}

\item $\mathcal{NP}\text{-complete}$ ne\' cemo definisati formalno.
\item Problemi iz skupa $\mathcal{NP}$ mogu se redukovati (transformisati) jedan u drugi. 

\item Od značaja su transformacije koje se mogu izvesti u polinomskom vremenu.

\item Na taj način problem A može da se svede na ekvivalentan problem B.  Ako smo u stanju da rešimo problem B, to rešenje može da se brzo adaptira i iskoristi i za rešavanje problema A.

\item Za problem kažemo da je $\mathcal{NP}$-kompletan (“$\mathcal{NP}$-Complete”):
\begin{itemize}
\item ako je u pitanju problem odlučivosti (rezultat se svodi na “da” ili “ne”);

\item ako se rešenje problema može proveriti “brzo”, u polinomskom vremenu (sudoku, rubikova kocka)

\item ako problem može da se iskoristi za rešavanje bilo kog drugog problema iz skupa $\mathcal{NP}$.
\end{itemize}

\item Do danas je definisano preko 1.000 problema koji spadaju u $\mathcal{NP}$-kompletan skup. Svi $\mathcal{NP}$ problemi svode se na neki od njih.

\item $\mathcal{NP}$-kompletni problemi su kao domine: dovoljno je rešiti samo jedan $\mathcal{NP}$-kompletan problem u polinomskom vremenu pa da svi $\mathcal{NP}$ problemi budu takođe rešeni u polinomskom vremenu.
\end{itemize}

\end{frame}

\begin{frame}{$\mathcal{NP}$-hard klasa problema}
\begin{itemize}
\item Ni ovu klasu ne defini\v semo formalno.
\item $\mathcal{NP}$-teški problemi u najvećem broju slučajeva predstavljaju komplementarnu, funkcionalnu verziju problema odlučivosti koji spadaju u $\mathcal{NP}$ ili $\mathcal{NP}$-kompletne probleme. Kažemo da ovi problemi nastaju uopštavanjem (generalizacijom tj. redukcijom) $\mathcal{NP}$ problema u polinomskom vremenu.

\item $\mathcal{NP}$-teški problemi mogu biti i odlučivog i funkcionalnog tipa.

\item Kažemo da su ovi problemi „teški“ za rešavanje koliko i najteži problemi iz $\mathcal{NP}$ skupa. Čak je i okvirna procena rešenja teška ili nemoguća.

\item Za razliku od $\mathcal{NP}$ problema čije rešenje mora biti proverljivo u polinomskom vremenu, $\mathcal{NP}$-teški problemi nemaju to ograničenje.

\item U $\mathcal{NP}$-tešku grupu problema spadaju i „neodlučivi problemi“, tj. problemi za koje ne postoji algoritamsko rešenje.

\item Mnogi $\mathcal{NP}$-kompletni problemi imaju svoju $\mathcal{NP}$-tešku verziju (problem najveće klike, problem sume podskupa, izračunavanje hromatskog broj grafa). Primere \' cemo videti u nastavku.
\end{itemize}
\end{frame}

\section{Primeri $\mathcal{P}$ i $\mathcal{NP}$ problema}

\begin{frame}{Primeri nekih $\mathcal{P}$ problema}

\begin{itemize}
    \item<1-> Tra\v zenje elementa u nizu - $O(n)$.
    \item<2-> Tra\v zenje elementa u sortiranom nizu (binarna pretraga) - $O(\log n)$.
    \item<3-> Provera da li je broj prost - $O(\sqrt{n})$.
    \item<4-> Tra\v zenje medijane u nizu - $O(n \log n)$.
    \item<5-> Mno\v zenje kvadratnih matrica - $O(n^3)$
\end{itemize}

\end{frame}

\begin{frame}{Problem faktorizacije broja}
    \begin{itemize}
        \item Fundamentalna teorema teorije brojeva ka\v ze da se svaki broj mo\v ze na jedistven na\v cin napisati kao proizvod prostih brojeva (faktorizovati).
        \item Na primer, $108 = 2^2 \cdot 3^3$.
        \item Za sada nije poznat nijedan algoritam koji ovaj problem re\v sava u polinomskom vremenu.
        \item Tako\dj e nije poznato da li je ovaj problem $\mathcal{NP}$-kompletan, te se njegovim re\v savanjem u polinomskom vremenu ne bi pokazalo da va\v zi $\mathcal{P} = \mathcal{NP}$. Pretpostavlja se da ovaj problem pripada posebnoj klasi $\mathcal{NP}$-intermediate.
        \item Za sada, najbolji poznati algoritam ima slo\v zenost $O\left( \left(1 + \varepsilon \right)^b\right)$ gde je $\varepsilon$ neka vrlo mala konstanta a $b$ broj bitova koji se faktorizuje.
        \item Na ovom algoritmu se zasniva ogroman deo internet kriptografije ($RSA$-algoritam). Zasad nisu prona\dj ene mane ovog algoritma.
    \end{itemize}
\end{frame}

\begin{frame}{SAT problem (Boolean satisfiability problem) }
    \begin{itemize}
        \item \textbf{SAT} je prvi problem za koji je dokazano da je $\mathcal{NP}$-kompletan. U svom revoucionarnom radu iz 1971. godine \textbf{Stiven Kuk} i \textbf{Leonid Levin} su dokazali da se svaki $\mathcal{NP}$ problem mo\v ze svesti na SAT u polinomskom vremenu (Pogledati \cite{cook}). 
        \item Sat problem se odnosi na logi\v cke izraze u kojima mo\v ze figurisati proizvoljan broj \emph{true/false} promenljivih i logi\v ckih operatora. Na primer:
        $$\left( (x_1 \lor \neg x_2) \Rightarrow (x_3 \land \neg x_4) \right) \land \left( \neg (x_1 \land x_3) \lor x_2 \right)$$
        
        \item \textbf{Odlu\v civa verzija problema:} Da li se svakoj promenjljivoj mo\v ze dodeliti vrednost tako da logi\v cki izraz na kraju ima vrednost \emph{true}? Algoritam grube sile ima slo\v zenost $O(2^n)$ gde je n broj promenjljivih u izrazu.
        \item Postoji i upor\v s\' cena verzija problema koja posmatra samo izraze u $KNF$ \v ciji konjunkti imaju tri disjunkta. Ova verzija se zove \textbf{3SAT} i tako\dj e je $\mathcal{NP}$-kompletna. Primer izraza koje posmatra 3SAT:
        $$(x_1 \lor \neg x_2 \lor x_3) \land (\neg x_1 \lor x_2 \lor x_4) \land (x_3 \lor \neg x_4 \lor x_5)$$
    \end{itemize}
\end{frame}

\begin{frame}{Problem trgova\v ckog putnika (Travelling salesman)}


\noindent\begin{minipage}{0.6\textwidth}
    \leftflush
    \textbf{Formulacija problema: Na\' ci najkra\' ci zatvoreni put koji kroz svako zadato mesto prolazi samo jednom.}
    \begin{itemize}
        \item U svojoj odlu\v civoj verziji (“Da li postoji zatvoreni put du\v zine manje od $k$?”), problem je $\mathcal{NP}$-kompletan.
        \item Originalni problem je  $\mathcal{NP}$-te\v zak!.
        \item Najbolji poznat algoritam za re\v savanje $\mathcal{NP}$-te\v skog problema je slo\v zenosti $O(n^2 2^n)$. Naivni algoritam grube sile (proveravanje svih permutacija) ima slo\v zenost $O(n!)$. 
        \item Problem se sre\' ce svuda: u planiranju, logistici, konstrukciji mikroprocesora i mnogim problemima optimizacije. Grana primenjene matematike koja se bavi pitanjima sli\v cnim ovom problemu se naziva operaciona istra\v zivanja.
    \end{itemize}
\end{minipage}%
\hfill
\begin{minipage}{0.38\textwidth}
    \raggedleft
    \includegraphics[width=\linewidth]{Travelling Salesman - Wolfram.png}
\end{minipage}

\end{frame}

\section{Zaklju\v cak}

\begin{frame}{Implikacije $\mathcal{P}$ vs $\mathcal{NP}$ problema}

\begin{itemize}
\item Postoje dva moguća dokaza za P=NP, konstruktivan i nekonstruktivan
\item U slučaju konstruktivnog dokaza svet bi bio potpuno drugačije mesto.
\item U slučaju nekonstruktivnog dokaza (verovatniji scenario) sve bi se odvijalo po starom
\item Većina matematičara danas veruje da je P različito od NP.
\item \textbf{Donald Knuth:} “My main point, however, is that I don't believe that the equality P = NP will turn out to be helpful even if it is proved, because such a proof will almost surely be nonconstructive.”
\end{itemize}

\begin{figure}
    \centering
    \includegraphics[width=0.20\linewidth]{DonaldKnuth.jpg}
    \caption{Enter Caption}
\end{figure}

\end{frame}

\begin{frame}{Literatura}
    \begin{thebibliography}{9}
    
    \bibitem{hopcroft}
    J. E. Hopcroft, R. Motwani, and J. D. Ullman, 
    \textit{Introduction to Automata Theory, Languages, and Computation}, 
    3rd Edition, Addison-Wesley, 2006.
    
    \bibitem{cook}
    S. Cook, 
    \textit{The Complexity of Theorem-Proving Procedures}, 
    Proceedings of the Third Annual ACM Symposium on Theory of Computing (STOC), 1971, pp. 151–158.
    
    \end{thebibliography}
\end{frame}
    


\end{document}
