\documentclass[aspectratio=169, xcolor=table, 10pt]{beamer}
\usepackage[no-math]{fontspec}
\usepackage[compatibility=false]{caption}
\usepackage{makecell}
\usepackage{ragged2e}
\usepackage{multicol}
% Thin border globally
\setlength{\fboxrule}{0.5pt} % border thickness
\setlength{\fboxsep}{0.5pt}    % padding inside border

\captionsetup{labelformat=empty}
\setmainfont{MinionPro-Regular.otf}[
  Path = font/,
  BoldFont = MinionPro-Bold.otf,
  ItalicFont = MinionPro-It.otf,
  BoldItalicFont = MinionPro-BoldIt.otf
]
\setsansfont{MinionPro-Regular.otf}[
  Path = font/,
  BoldFont = MinionPro-Bold.otf,
  ItalicFont = MinionPro-It.otf,
  BoldItalicFont = MinionPro-BoldIt.otf
]
\newfontfamily\cyrillicfontsf{MinionPro-Regular.otf}[
  Path = font/,
  BoldFont = MinionPro-Bold.otf,
  ItalicFont = MinionPro-It.otf,
  BoldItalicFont = MinionPro-BoldIt.otf,
  Script = Cyrillic
]
\usepackage[utf8]{inputenc}
\usepackage{polyglossia}
\setmainlanguage{serbian}
\setotherlanguage{english}
\renewcommand{\baselinestretch}{1.1}
\usepackage{amsmath}
\usepackage{hyperref}
\usepackage{amsfonts}
\usepackage{amssymb}
\usepackage{amsthm}
\usepackage{graphicx}
\usepackage{tikz}
\usetheme{Madrid}
\usecolortheme{crane}

\setbeamercolor{table even}{bg=yellow} % Yellow for even rows
\setbeamercolor{table odd}{bg=yellow}  % Yellow for odd rows
\setbeamercolor{last row}{bg=orange}   % Orange for the last row

\theoremstyle{definition}
\newtheorem{defi}{Дефиниција}[section]

\newtheorem{prim}{Пример}[section]

\AtBeginSection[]
{
  \begin{frame}
    \frametitle{Садржај}
    \tableofcontents[currentsection]
  \end{frame}
}

\title{Херцшпрунг-Раселов дијаграм}
\author{Лука Марковић}
\institute[МАТФ]{Математички факултет Универзитета у Београду}
\date{Септембар 2025.}

\begin{document}

\maketitle

\section{Класификација звезда према површинској температури}

\begin{frame}{Какве све звезде постоје?}
  \begin{itemize}
    \item Не постоје две идентичне звезде.
    \item Звезде се разликују по маси, величини, површинској температури, боји, сјају и количини енергије коју емитују у јединици времена.
    \item Да ли је свака комбинација ових параметара могућа?
    \item Нека ограничења сигурно постоје: не постоје зелене или љубичасте звезде.
    \item Везе између наведених величина морају бити ”апсолутне”, независне од места посматрача.
    \item Нека својства звезда могуће је лако утврдити: типичан пример је боја звезде.
    \item Нека друга својства је много теже измерити.
    \item Тако, на пример, количина енергије коју звезда емитује зависи од њене величине и апсолутног сјаја.
    \item Са Земље, међутим, лако можемо да утврдимо само привидни сјај звезде.
    \item Привидни сјај звезде зависи од апсолутног сјаја али и од растојања на коме се звезда налази.
  \end{itemize}
\end{frame}

\begin{frame}{Класе звезда}
  \begin{itemize}
    \item Све звезде деле се на класе према тзв. Морган-Кинановом систему разрађеном на Харварду почетком XX века.
    \item Класа се одређује мерењем ширине каралтеристичних Фраунхоферових линија водоника, хелијума, калцијума и титанијум-оксида у апсорпционом спектру звезде.
    \item Што је звезда топлија, ширина Фраунхоферових линија је већа.
    \item Прво су дефинисане класе \textbf{O}, \textbf{B}, \textbf{A}, \textbf{F}, \textbf{G}, \textbf{K} i \textbf{M}.
    \item Површинска температура звезде опада од класе \textbf{O} (најтоплије звезде) до \textbf{M} (најхладније).
    \item Свака класа има десет поткласа (0-9) у зависности од температуре. Сунце се налази у класи \textbf{G2}.
    \item Касније су додате класе \textbf{D} (бели патуљци), \textbf{L} i \textbf{T} (смеђи патуљци), \textbf{S} i \textbf{C} (угљеничне звезде).
    \item Овај систем користи се и даље али постоје и други, прецизнији, квантитативни системи.
  \end{itemize}
\end{frame}

\begin{frame}{Графички приказ звезданих класа}
  \begin{figure}
    \centering
    \sbox0{\fbox{\includegraphics[height=0.65\textheight]{images/klase_zvezda.jpg}}}
    \usebox0
    \captionsetup{width=\wd0}
    \caption{Главне спектралне класе са одговарајућим температурама, апроксимативним бојама и релативним величинама звезде}
  \end{figure}
\end{frame}

\begin{frame}{Индекс температуре звезде $B-V$}
  \begin{itemize}
    \item $B-V$ индекс одређује се посматрањем сјаја звезде кроз два филтера.
    \item $B$-индекс одређује се посматрањем сјаја звезде кроз $UB$-филтер осетљив на ултраљубичасту и плаву боју.
    \item $V$-индекс одређује се посматрањем сјаја звезде кроз $VB$-филтер осетљив на већи део видљивог спектра (плаву, жуту и зелену боју).
    \item $B-V$ индекс представља разлику ова два индекса. Систем је тако баждарен да звезда Вега има $B-V$ индекс једнак нули.
    \item Мање вредности индекса одговарају топлијим звездама, веће вредности - хладнијим.
    \item $B-V$ индекс за Сунце износи 0,656, а за плавичасти Ригел -0,03.
    \item Из овог индекса може се директно израчунати температура површине звезде:
  \end{itemize}
  \begin{equation*}
  T=4600\text{K}\cdot\left[\frac{1}{0,92(B-V)+1,7}+\frac{1}{0,92(B-V)+0,62}\right]
  \end{equation*}
\end{frame}

\begin{frame}{Детаљи спектралне класификације}
  \begin{itemize}
    \item Класа звезде, њена температурa и индекс боје су три еквивалентне величине.
  \end{itemize}
  \begin{table}
    \vspace{-1em}
    \resizebox{\textwidth}{!}{
      \begin{tabular}{|c|c|c|c|c|>{\RaggedRight}p{4cm}|c|}
      \hline
      \rowcolor{red} \textbf{\textcolor{white}{Класа}} & \textbf{\textcolor{white}{Температура (K)}} & \textbf{\textcolor{white}{Боја}} & \textbf{\textcolor{white}{Маса ($M_\odot)$}} & \textbf{\textcolor{white}{\makecell{Водоничне линије \\ спектра}}} & \textbf{\textcolor{white}{\makecell[l]{Друге линије \\ спектра}}} & \textbf{\textcolor{white}{\makecell{Проценат \\ звезда}}} \\
      \hline
      \hline
      \rowcolor{red!10}\textbf{O} & > 33.000 & плава & > 16 & слабе & вишеструко јонизовани атоми & ~0,0003\% \\
      \hline
      \rowcolor{red!20}\textbf{B} & 10.000 - 33.000 & плава/бела & 2,1 - 16 & средње & неутрални хелијум & ~0,1\% \\
      \hline
      \rowcolor{red!10}\textbf{A} & 7.500 - 10.000 & бела/плава & 1,4 - 2,1 & јаке & јонизовани калцијум, неутрални хелијум & ~0,6\% \\
      \hline
      \rowcolor{red!20}\textbf{F} & 6.000 - 7.500 & бела & 1,0 - 1,4 & средње & јонизовани калцијум, метали & ~3\% \\
      \hline
      \rowcolor{red!10}\textbf{G} & 5.200 - 6.000 & жута & 0,8 - 1,0 & слабе & јаке линије метала & ~7,5\% \\
      \hline
      \rowcolor{red!20}\textbf{К} & 3.700 - 5.200 & жута/оранж & 0,45 - 0,8 & врло слабе & неутрални калцијум, титан-оксид & ~12\% \\
      \hline
      \rowcolor{red!10}\textbf{М} & < 3.700 & оранж/црвена & < 0,45 & врло слабе & јаке линије калцијума и титан-оксида, неутрални метали & ~76\% \\
      \hline
      \end{tabular}
    }
  \end{table}
\end{frame}

\section{Луминозност, привидни и апсолутни сјај звезде}

\begin{frame}{Привидни сјај}
  \begin{columns}[T]
    \begin{column}{0.35\textwidth}
      \begin{figure}
        \centering
        \sbox0{\fbox{\includegraphics[height=0.60\textheight]{images/hiparh.jpg}}}
        \usebox0
        \captionsetup{width=\wd0}
        \caption{Хипарх, један од највећих античких астронома}
      \end{figure}
    \end{column}
    \begin{column}{0.65\textwidth}
      \begin{itemize}
        \item Довољан је један поглед на ноћно небо па да утврдимо да звезде сјаје различитим интензитетом.
        \item Хипарх је 135. године п.н.е направио каталог са 850 звезда у коме је свакој звезди доделио број од 1 до 6, у зависности од сјаја (магнитуде).
        \item Најсјајније звезде имале су ознаку 1 док су оне једва видљиве имале ознаку 6.
        \item Развој телескопа и фотометрије омогућио је да се сјај звезда објективно измери.
        \item У исто време утврђено је да људско око детектује промене у интензитету светлости по логартиамској скали.
        \item Такође је утврђено да звезде са ознаком 1 имају 100 пута мањи сјај од звезде са ознаком 6.
      \end{itemize}
    \end{column}
  \end{columns}
\end{frame}

\begin{frame}{Савремена скала звезданих магнитуда}
  \begin{itemize}
    \item Хипархов систем, уз незнатне измене, користи се и данас!
    \item Суштински, користи се чињеница да, према Хипарховој логаритамској скали, разлика у магнитуди величине 5 одговара 100 пута јачем или слабијем интензитету светлости.
    \item То значи да је звезда која има магнитуду за један број мању од друге сјајнија од ње $\sqrt[5]{100}\approx2,5$ пута.
    \item Хипархов у међувремену je рекалибрисан:
    \begin{itemize}
      \item Звезда Вега има магнитуду 0.
      \item Задржано је старо емпиријско правило да 100 пута сјајнија (тамнија) звезда има магнитуду мању (већу) за 5.
    \end{itemize}
  \item С обзиром да на небу постоје објекти сјајнији од Веге (Сунце, Месец, Венера, Сиријус...), њихова магнитуда је негативна!
  \item Овако измерене магнитуде називају се \textbf{привидним} јер зависе од растојања посматраног објекта од нас.
  \end{itemize}
\end{frame}

\begin{frame}{Сјај неких небеских тела}
    \begin{multicols}{2}
    \begin{itemize}
      \item Сунце: -26,5
      \item Пун Месец: -12,5
      \item Венера: -4,3
      \item Марс и Јупитер: -3
      \item Меркур: -2
      \item Сиријус: -1,44
      \item Вега, Сатурн: 0
      \item Антарес: 1
      \item Северњача: 2
      \item Уран: 5
      \item Лимит голог ока: 6,5
      \item Церес: 7
      \item Нептун: 8
      \item Лимит двогледа: 10
      \item Проксима кентаури: 11,1
      \item Плутон: 14
      \item Лимит телескопа (8m): 28
      \item Лимит телескопа ”Хабл”: 32
    \end{itemize}
  \end{multicols}

\end{frame}


\end{document}
